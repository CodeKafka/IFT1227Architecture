\documentclass{report}
\usepackage{multirow}
\usepackage{array}
\input{preamble.tex}
\usepackage{circuitikz}
\usepackage[scr]{rsfso}
\usepackage{microtype}


%===============================
\usetikzlibrary{circuits.ee.IEC}
\usetikzlibrary{arrows,shapes.gates.logic.US,shapes.gates.logic.IEC,calc, positioning}
\newcommand{\faketarget}{\oplus\!\!\!\!\odot}


\titleformat{\chapter}[display]
  {\normalfont\bfseries\color{black}}
  {\filleft%
    \begin{tikzpicture}
    \node[
      outer sep=0pt,
      text width=1.5cm,
      minimum height=2cm,
      fill=draculawhite-background,
      font=\color{black}\fontsize{40}{50}\selectfont,
      align=center
      ] (num) {\thechapter};
    \node[
      rotate=90,
      anchor=south,
      font=\color{black}\Large\normalfont
      ] at ([xshift=-5pt]num.west) {\textls[180]{\textsc{\textit{Section}}}};  
    \end{tikzpicture}%
  }
  {5pt}
  {\titlerule[2.0pt]\vskip3pt\titlerule\vskip4pt\large\normalfont}



\titleformat{\section}
  {\normalfont\scshape}{\thesection}{1em}{}

\titleformat{\subsection}
  {\normalfont\scshape}{\thesubsection}{1em}{}


% Customizing the spacing for the chapter titles
\titlespacing*{\chapter}{0pt}{0pt}{20pt}



\DeclareRobustCommand{\looongrightarrow}{%
  \DOTSB\relbar\joinrel\relbar\joinrel\relbar\joinrel\rightarrow
}


\DeclareMathOperator{\di}{d\!}
\newcommand*\Eval[3]{\left.#1\right\rvert_{#2}^{#3}}




\title{\Huge{Architecture des ordinateurs}\\{IFT1227}\\{\textbf{Introduction}}}
\author{\huge{Franz Girardin}}
\date{\today}
\lstset{inputencoding=utf8/latin1}

            %%%%%%%%%%%%%%%%%  Sect.       %%%%%%%%%%%%%%%%%%%%%%%%%%%%%%%%%%%%%%%%%%%%%%%%%%%%%%%%%







\begin{document}
\maketitle

\pagebreak

\pagebreak
\begin{multicols*}{3}


    \footnotesize
    \chapter{Circuits logiques}


    \paragraph{Couche logique numérique}  
    Constituées de \textit{portes logiques} construite à partir 
    de \textbf{transisteurs} qui prennent un \textbf{signal}
    \textbf{0} ou \textbf{1} et calcule 
    une \textbf{fonction logique} 
    $\mathbb{ET}$, $\mathbb{OU}$ et $\mathbb{NON}$, etc.

    \paragraph{Porte $\mathbb{NON}$}
    \mbox{}\vspace{1em}\\
    \begin{minipage}{\columnwidth}
        \begin{minipage}[b]{0.5\columnwidth}
            \centering
            \notgate
        \end{minipage}%
        \begin{minipage}[b]{0.5\columnwidth}
            \centering
            \raisebox{0.3\height}{
            \renewcommand{\arraystretch}{1.5}
            \begin{tabular}{c|c}
                A & Y \\
                \hline
                1 & 0 \\
                0 & 1 \\
            \end{tabular}
            }
        \end{minipage}
    \end{minipage}
    \[\entouree{$Y= \overline{A}$} \]


    \paragraph{Porte $\mathbb{BUF}$}
    \mbox{}\vspace{1em}\\
    \begin{minipage}{\columnwidth}
        \begin{minipage}[b]{0.5\columnwidth}
            \centering
            \bufgate
        \end{minipage}%
        \begin{minipage}[b]{0.5\columnwidth}
            \centering
            \raisebox{0.3\height}{
            \renewcommand{\arraystretch}{1.5}
            \begin{tabular}{c|c}
                A & Y \\
                \hline
                0 & 0 \\
                1 & 1 \\
            \end{tabular}
            }
        \end{minipage}
    \end{minipage}
    \[\entouree{$Y= A$} \]



    \paragraph{Porte $\mathbb{ET}$}
    \mbox{}\vspace{1em}\\
    \begin{minipage}{\columnwidth}
        \begin{minipage}[b]{0.5\columnwidth}
            \centering
            \andgate
        \end{minipage}%
        \begin{minipage}[b]{0.5\columnwidth}
            \centering
            \raisebox{-0.3\height}{
            \renewcommand{\arraystretch}{1.5}
            \begin{tabular}{c c | c}
                A & B & Y \\
                \hline
                0 & 0 & 0 \\
                0 & 1 & 1 \\
                1 & 0 & 1 \\
                1 & 1 & 1 
            \end{tabular}
            }
        \end{minipage}
    \end{minipage}
    \[\entouree{$Y = AB $} \]



    \paragraph{Porte $\mathbb{OU}$}
    \mbox{}\vspace{1em}\\
    \begin{minipage}{\columnwidth}
        \begin{minipage}[b]{0.5\columnwidth}
            \centering
            \orgate
        \end{minipage}%
        \begin{minipage}[b]{0.5\columnwidth}
            \centering
            \raisebox{-0.3\height}{
            \renewcommand{\arraystretch}{1.5}
            \begin{tabular}{c c | c}
                A & B & Y \\
                \hline
                0 & 0 & 0 \\
                0 & 1 & 0 \\
                1 & 0 & 0 \\
                1 & 1 & 1 
            \end{tabular}
            }
        \end{minipage}
    \end{minipage}
    \[\entouree{$Y = A + B $} \]



    \paragraph{Porte $\mathbb{XOR}$}
    \mbox{}\vspace{1em}\\
    \begin{minipage}{\columnwidth}
        \begin{minipage}[b]{0.5\columnwidth}
            \centering
            \xorgate
        \end{minipage}%
        \begin{minipage}[b]{0.5\columnwidth}
            \centering
            \raisebox{-0.3\height}{
            \renewcommand{\arraystretch}{1.5}
            \begin{tabular}{c c | c}
                A & B & Y \\
                \hline
                0 & 0 & 0 \\
                0 & 1 & 1 \\
                1 & 0 & 1 \\
                1 & 1 & 0 
            \end{tabular}
            }
        \end{minipage}
    \end{minipage} 
    \[\entouree{$Y = A \oplus B$} \]



    \paragraph{Porte $\mathbb{NAND}$}
    \mbox{}\vspace{1em}\\
    \begin{minipage}{\columnwidth}
        \begin{minipage}[b]{0.5\columnwidth}
            \centering
            \nandgate
        \end{minipage}%
        \begin{minipage}[b]{0.5\columnwidth}
            \centering
            \raisebox{-0.3\height}{
            \renewcommand{\arraystretch}{1.5}
            \begin{tabular}{c c | c}
                A & B & Y \\
                \hline
                0 & 0 & 1 \\
                0 & 1 & 1 \\
                1 & 0 & 1 \\
                1 & 1 & 0 
            \end{tabular}
            }
        \end{minipage}
    \end{minipage}
    \[\entouree{$Y = \overline{AB}$} \]


    \paragraph{Porte $\mathbb{NOR}$}
    \mbox{}\vspace{1em}\\
    \begin{minipage}{\columnwidth}
        \begin{minipage}[b]{0.5\columnwidth}
            \centering
            \norgate
        \end{minipage}%
        \begin{minipage}[b]{0.5\columnwidth}
            \centering
            \raisebox{-0.3\height}{
            \renewcommand{\arraystretch}{1.5}
            \begin{tabular}{c c | c}
                A & B & Y \\
                \hline
                0 & 0 & 1 \\
                0 & 1 & 0 \\
                1 & 0 & 0 \\
                1 & 1 & 0 
            \end{tabular}
            }
        \end{minipage}
    \end{minipage}
    \[\entouree{$Y = \overline{A + B}$} \]



    \paragraph{Porte $\mathbb{XNOR}$}
    \mbox{}\vspace{1em}\\
    \begin{minipage}{\columnwidth}
        \begin{minipage}[b]{0.5\columnwidth}
            \centering
            \xnorgate
        \end{minipage}%
        \begin{minipage}[b]{0.5\columnwidth}
            \centering
            \raisebox{-0.3\height}{
            \renewcommand{\arraystretch}{1.5}
            \begin{tabular}{c c | c}
                A & B & Y \\
                \hline
                0 & 0 & 1 \\
                0 & 1 & 0 \\
                1 & 0 & 0 \\
                1 & 1 & 1 
            \end{tabular}
            }
        \end{minipage}
    \end{minipage}
    \[\entouree{$Y = \overline{A \oplus B}$} \]



    \paragraph{Porte $\mathbb{NOR}3$}
    \mbox{}\vspace{1em}\\
    \begin{minipage}{\columnwidth}
        \begin{minipage}[b]{0.5\columnwidth}
            \centering
            \northreegate
        \end{minipage}%
        \begin{minipage}[b]{0.5\columnwidth}
            \centering
            \raisebox{-0.6\height}{
            \renewcommand{\arraystretch}{1.5}
            \begin{tabular}{c c c | c}
                A & B & C & Y \\
                \hline
                0 & 0 & 0 & 1 \\
                0 & 0 & 1 & 0 \\
                0 & 1 & 0 & 0 \\
                0 & 1 & 1 & 0 \\
                1 & 0 & 0 & 0 \\
                1 & 0 & 1 & 0 \\
                1 & 1 & 0 & 0 \\
                1 & 1 & 1 & 0 
            \end{tabular}
            }
        \end{minipage}
    \end{minipage}
    \[\entouree{$Y = \overline{A + B + C}$} \]
    \columnbreak



    \paragraph{Porte $\mathbb{AND}3$}
    \mbox{}\vspace{1em}\\
    \begin{minipage}{\columnwidth}
        \begin{minipage}[b]{0.5\columnwidth}
            \centering
            \andthreegate
        \end{minipage}%
        \begin{minipage}[b]{0.5\columnwidth}
            \centering
            \raisebox{-0.6\height}{
            \renewcommand{\arraystretch}{1.5}
            \begin{tabular}{c c c | c}
                A & B & C & Y \\
                \hline
                0 & 0 & 0 & 0 \\
                0 & 0 & 1 & 0 \\
                0 & 1 & 0 & 0 \\
                0 & 1 & 1 & 0 \\
                1 & 0 & 0 & 0 \\
                1 & 0 & 1 & 0 \\
                1 & 1 & 0 & 0 \\
                1 & 1 & 1 & 1 
            \end{tabular}
        }         
    \end{minipage}
    \end{minipage}
    \[\entouree{$Y = ABC$} \]

    \paragraph{Définition de la marge de bruit}
    Tolérance d'un circuit aux \textbf{perturbations} 
    pouvant fausser l'interprétation du signal.  
    \begin{itemize}
        \item[$\rhd$] $ V_{IH}$ : $\min(V | V_{in}\Coloneqq \textbf{1})$  
        \item[$\rhd$]  $V_{IL}$ : $\max(V | V_{in}\Coloneqq \textbf{0})$
        \item[$\blacktriangleright$] Signal reçu \textbf{min} ou \textbf{max} 
            est être interprété comme \textbf{1}  ou \textbf{0}. 
        \item[$\rhd$] $ V_{OH}$ : $\min(V | V_{out}\Coloneqq \textbf{1})$  
        \item[$\rhd$] $V_{OL}$ : $\max(V | V_{out}\Coloneqq \textbf{0})$
        \item[$\blacktriangleright$] Signal \textbf{min} ou \textbf{max} 
            que l'émetteur s'engage à fournir pour être interprété 
            comme \textbf{1}  ou \textbf{0}. 
    \end{itemize}
    \begin{align*}
        NM_H = V_{OH} \; (\textit{ém.}) - V_{IH} \; (src.) \\
        NM_L = V_{IL} (src.) - V_{OL} (\textit{ém.})
    \end{align*}

    \begin{figure}[H]
        \begin{center}
            \includegraphics[width=0.30\textwidth]{MargeBruit.png}
        \end{center}
    \end{figure}

    \paragraph{Transistors}
    \mbox{}\vspace{1em}\\
    Éléments de base des circuits électroniques. Ils sont composés de 
    trois broches; le drain, la source, et la \textbf{grille} qui contrôle 
    les deux autres comme un interrupteur.  La figure suivante représente 
    un transistor hors tension \textit{off}, un transistor hors tension 
    mais polarisé \textit{off} et un transistor sous tension \textit{on}.     
    \begin{center}
    \begin{circuitikz}[scale=0.5]
    % Draw NMOS transistor
    \draw (0,0) node[nigfete] (mos) {}
    (mos.gate) node[left] {g}
    (mos.drain) node[above] {d}
    (mos.source) node[below] {s};

    % Draw NMOS in OFF state
    \draw (3,0) node[nigfete] (mosoff) {}
    (mosoff.gate) node[left] {g}
    (mosoff.drain) node[above] {d}
    (mosoff.source) node[below] {s};
    \draw (mosoff.gate) -- ++(0.5,0) node[circ] {} -- (mosoff.source);

    Draw NMOS in ON state
    \draw (6,0) node[nigfete] (moson) {}
    (moson.gate) node[left] {g}
    (moson.drain) node[above] {d}
    (moson.source) node[below] {s};
    \draw (moson.gate) -- ++(0.5,0) node[circ] {} -- (moson.drain);
    \end{circuitikz}
    \end{center}

    \paragraph{Composition d'un circuit} 
    \mbox{}\\ 
    \textbf{Circuit}$\Coloneqq$ E., S., spec. \textit{fonct}., spec. \textit{temp}.   

    \paragraph{Propriétés d'un circ. combinatoire}
    \mbox{} 
    \vspace{0.1em}
    \begin{itemize}
        \item[$\rhd$]  \textbf{Noeud} $\implies$ \textit{In}  
            ou \textbf{connexion à} un \textit{Out}. 
        \item[$\rhd$] \textbf{Aucun} chemin cyclique.    

    \end{itemize}


    \paragraph{Sommes de produits SOP}
    En considérant les variables \textit{In} d'une ligne de la table de vérité, 
    il faut identifier la \textbf{conjonction} (produit) nécessaire 
    pour engendrer un \textbf{1} logique. Un \textbf{minterm} est une 
    représentation du produit engendrant un 1 logique. 

                \begin{table}[H]
                  \begin{center}
                   \renewcommand{\arraystretch}{1.5}
                    \footnotesize
                        \begin{tabular}{|l|l|l||c|}
                        \arrayrulecolor{blue}\hline
                        \rowcolor{lightBlue}
                        \textcolor{myb}{$A$} & \textcolor{myb}{$B$} 
                                           & \textcolor{myb}{$Y$} 
                                           & \textcolor{myb}{\textit{minterm}}  
                        \\
                        \hline
                        \hline
                        \arrayrulecolor{black}
                        $V$ & $F$ & \cellcolor{myr} $0$ & $\overline{AA}$
                        \\
                        \hline
                        $0$ & $1$ & \cellcolor{myg} $1$ & \textcolor{myg}{$\overline{A}B$}
                        \\
                        \hline 
                        $1$ & $0$ & \cellcolor{myr} $0$ & $A\overline{B}$ 
                        \\ 
                        \hline
                        $1$ & $1$ & \cellcolor{myg} $1$ & \textcolor{myg}{$AB$}
                        \\
                        \hline
                        \end{tabular}
                \end{center}
                \end{table}
                \[Y(A, B) = \overline{A}B + AB \leftrightarrow Y(A,B) = \sum (1, 3) \]
    

    \paragraph{Produits de sommes}
    En considérant les variables \textit{In} d'une ligne de la table de vérité, 
    il faut identifier la \textbf{somme} nécessaire 
    pour engendrer un \textbf{0} logique. Un \textbf{maxterm} est une 
    représentation de la somme engendrant un 0 logique. 


                \begin{table}[H]
                  \begin{center}
                   \renewcommand{\arraystretch}{1.5}
                     \footnotesize
                        \begin{tabular}{|l|l|l||c|}
                        \arrayrulecolor{blue}\hline
                        \rowcolor{lightBlue}
                        \textcolor{myb}{$A$} & \textcolor{myb}{$B$} 
                                           & \textcolor{myb}{$Y$} 
                                           & \textcolor{myb}{\textit{maxterm}}  
                        \\
                        \hline
                        \hline
                        \arrayrulecolor{black}
                        $V$ & $F$ & \cellcolor{myg} $0$ & \textcolor{myg}{$A + B$}
                        \\
                        \hline
                        $0$ & $1$ & \cellcolor{myr} $1$ & $A + \overline{B}$
                        \\
                        \hline 
                        $1$ & $0$ & \cellcolor{myg} $0$ & \textcolor{myg}{$\overline{A} + B$} 
                        \\ 
                        \hline
                        $1$ & $1$ & \cellcolor{myr} $1$ & $\overline{A} + \overline{B}$ 
                        \\
                        \hline
                        \end{tabular}
                \end{center}
                \end{table}
                \[Y(A, B) = (A+B)(A + \overline{B}) = \prod (0, 2) \]



    \begin{table}[H]
      \centering
      \renewcommand{\arraystretch}{1.5}
      \setlength{\arrayrulewidth}{0.4pt}
      \arrayrulecolor{blue}
      \scriptsize
      \begin{tabular}{|l|l|l|}
        \hline
        \rowcolor{lightBlue}
        \textcolor{myb}{Axiome} & \textcolor{myb}{Dual} & \textcolor{myb}{Nom} \\
        \hline
        \hline
        \( B = 0 \) if \( B \neq 1 \) & \( B = 1 \) if \( B \neq 0 \) & Bin. field \\
        \rowcolor{lightBlue}
        \( \overline{0} = 1 \) & \( \overline{1} = 0 \) & NOT \\
        \( 0 \cdot 0 = 0 \) & \( 1 + 1 = 1 \) & AND/OR \\
        \rowcolor{lightBlue}
        \( 1 \cdot 1 = 1 \) & \( 0 + 0 = 0 \) & AND/OR \\
        \( 0 \cdot 1 \cdot 0 = 0 \) & \( 1 + 0 = 1 + 1 \) & AND/OR \\
        \hline
        \end{tabular}
    \end{table}



    \begin{table}[H]
      \centering
      \renewcommand{\arraystretch}{1.5}
      \setlength{\arrayrulewidth}{0.4pt}
      \arrayrulecolor{blue}
      \footnotesize
      \begin{tabular}{|l|l|l|}
        \hline
        \rowcolor{lightBlue}
        \textcolor{myb}{Théorème} & \textcolor{myb}{Dual} & \textcolor{myb}{Nom} \\
        \hline
        \hline
        $B \cdot 1 = B$ & $B + 0 = B$ & Identité \\
        \rowcolor{lightBlue}
        \( B \cdot  0 = 0\) & \( B + 1 = 1 \) & Élément nul \\
        \( B \cdot B = B \) & \( B + B = B \) & Indépotence \\
        \rowcolor{lightBlue}
                            & \( \overline{\overline{B}}  = B\) & Involution \\
        \( B \cdot \overline{B} = 0 \) & \( B + \overline{B} = 1 \) & Complément \\
        \hline
        \end{tabular}
    \end{table}


    \paragraph{De Morgan}
    \mbox{}\vspace{0.5em}
    \[ \neg(A + B) = \neg A + \neg B \]
    \[ \neg(A \cdot B) = \neg A \cdot \neg B \]

    \paragraph{Exemple simple de circuit combinatoire} 
    \mbox{}\vspace{0.5em}



\tikzstyle{branch}=[fill,shape=circle,minimum size=3pt,inner sep=0pt]
\begin{tikzpicture}[label distance=1mm, scale=0.75]

    \node (x0) at (0,0) {$A$};
    \node (x1) at (1,0) {$B$};
    \node (x2) at (2,0) {$C$};

    \node[not gate US, draw, rotate=-90] at ($(x2)+(0.5,-1)$) (Not2) {};
    \node[not gate US, draw, rotate=-90] at ($(x1)+(0.5,-1)$) (Not1) {};
    \node[not gate US, draw, rotate=-90] at ($(x0)+(0.5,-1)$) (Not0) {};


    \node[and gate US, draw, logic gate inputs=nnn] at ($(x2)+(2,-2)$) (And1) {};
    \node[and gate US, draw, logic gate inputs=nnn] at ($(And1)+(0,-1)$) (And2) {};
    \node[and gate US, draw, logic gate inputs=nnn] at ($(And2)+(0,-1)$) (And3) {};

    \node[or gate US, draw, rotate=-90, logic gate inputs=nnn] at ($(And3)+(1,-1)$) (Or1) {};


    \foreach \i in {0, 1, 2}
    {
        \path (x\i) -- coordinate (punt\i) (x\i |- Not\i.input);
        \draw (punt\i) node[branch] {} -| (Not\i.input);
    }




    \draw (Not0.output |- And1.input 1) node[branch] {} -- (And1.input 1);
    \draw (Not1.output |- And1.input 2) node[branch] {} -- (And1.input 2);
    \draw (Not2.output |- And1.input 3) node[branch] {} -- (And1.input 3);

    \draw (x0 |- And2.input 1) node[branch] {} -- (And2.input 1);
    \draw (Not1.output |- And2.input 2) node[branch] {} -- (And2.input 2);
    \draw (Not2.output |- And2.input 3) node[branch] {} -- (And2.input 3);


    \draw (x0 |- And3.input 1) node[branch] {} -- (And3.input 1);
    \draw (Not1.output |- And3.input 2) node[branch] {} -- (And3.input 2);
    \draw (x2 |- And3.input 3) node[branch] {} -- (And3.input 3);


     % Connect AND gates to OR gate
    \draw (And1.output) -| ([xshift=0.5cm]And1.output) -| (Or1.input 1);
    \draw (And2.output) -| ([xshift=0.5cm]And2.output) -| (Or1.input 2);
    \draw (And3.output) -| ([xshift=0.5cm]And3.output) -| (Or1.input 3);


    % Draw output from OR gate
    \draw (Or1.output)  node[below] {Y} ++(0, -0.5);

   % Additional branches for inputs
    \draw (x0) |- (And2.input 1);
    \draw (x0) |- (And3.input 1);

    \draw (Not0) |- (And1.input 1);
    \draw (Not1) |- (And1.input 2);
    \draw (Not2.output) |- (And1.input 3);


    \draw (Not1) |- (And2.input 2);
    \draw (Not2.output) |- (And2.input 3);


    \draw (x2) |- (And3.input 3);
    \draw (Not1) |- (And3.input 2);


    % Additional code to create a named coordinate at the branch point
    \path (x1) -- coordinate (branchB) (x1 |- Not1.input);

    % Now draw the line from B to the existing branch point
    \draw (x1) -- (branchB);

    \draw (And1.output) -- ++(1,0) node[right] (minterm1) {$\overline{A}\cdot\overline{B}\cdot\overline{C}$};
    \draw (And2.output) -- ++(1,0) node[right] (minterm2) {$A\overline{B}C$};
    \draw (And3.output) -- ++(1,0) node[right] (minterm3) {$AB\overline{C}$};
\end{tikzpicture}

    
    \[ Y = \overline{A}\cdot\overline{B}\cdot\overline{C} 
     + A\overline{B}\overline{C} + A\overline{B}C \]

     \paragraph{Simple règles de schématisation}
     Les \textbf{E}. sont en haut à gauche et les \textbf{S}. sont en bas à droite; 
     on utilise des \textbf{fils droits}, préférablement.   
    \vspace{1em}

\begin{center}
 \begin{tikzpicture}[node distance=1.5cm, scale=0.60]

    % Wire junction at a T
    \draw (0,0) -- (2,0);
    \draw (1,0) -- (1,-0.5);
    \node[below] at (1,-1) {(a) {Fils engendrant une jonction T}};

    % Wires connect at a dot
    \draw (0,-2) -- ++(2,0);
    \draw (1,-2) -- ++(0,-1);
    \filldraw (1,-2) circle (2pt);
    \node[below] at (1,-3) {(b) Connexion explicitée par un point};

    % Wires crossing without connecting
    \draw (0,-4) -- ++(2,0);
    \draw (1,-3.5) -- (1,-4.5);
    \node[below] at (0.75,-5) {(c) Fils croisés sans point (\;$\therefore$ non connectés)};

\end{tikzpicture}    
\end{center}

    \paragraph{Circuit de priorité}
    \mbox{}\vspace{0.5em}



\begin{center}
 \begin{tikzpicture}[scale=0.55]
    % Draw the box for the priority circuit
    \draw (0,0) rectangle (4,5);
    \node at (2,2.5) {\textit{Priorité}};

    % Draw the input lines and labels
    \draw (-1,4.5) -- (0,4.5) node[midway, above] {\(A_3\)};
    \draw (-1,3.5) -- (0,3.5) node[midway, above] {\(A_2\)};
    \draw (-1,2.5) -- (0,2.5) node[midway, above] {\(A_1\)};
    \draw (-1,1.5) -- (0,1.5) node[midway, above] {\(A_0\)};

    % Draw the output lines and labels
    \draw (4,4.5) -- (5,4.5) node[midway, above] {\(Y_3\)};
    \draw (4,3.5) -- (5,3.5) node[midway, above] {\(Y_2\)};
    \draw (4,2.5) -- (5,2.5) node[midway, above] {\(Y_1\)};
    \draw (4,1.5) -- (5,1.5) node[midway, above] {\(Y_0\)};
\end{tikzpicture}    
\end{center}


\begin{table}[H]
  \centering
  \renewcommand{\arraystretch}{1.5}
  \setlength{\arrayrulewidth}{0.4pt}
  \arrayrulecolor{blue}
  \scriptsize
  \begin{tabular}{|c|c|c|c||c|c|c|c|}
    \hline
    \rowcolor{lightBlue}
    \textcolor{myb}{$A_3$} & \textcolor{myb}{$A_2$} & \textcolor{myb}{$A_1$} & \textcolor{myb}{$A_0$} & \textcolor{myb}{$Y_3$} & \textcolor{myb}{$Y_2$} & \textcolor{myb}{$Y_1$} & \textcolor{myb}{$Y_0$} \\
    \hline
    \hline
    0 & 0 & 0 & 0 & 0 & 0 & 0 & 0 \\
    \rowcolor{lightBlue}
    0 & 0 & 0 & 1 & 0 & 0 & 0 & \textcolor{blue}{1} \\
    0 & 0 & 1 & 0 & 0 & 0 & \textcolor{blue}{1} & 0 \\
    \rowcolor{lightBlue}
    0 & 0 & 1 & 1 & 0 & 0 & \textcolor{blue}{1} & 0 \\
    0 & 1 & 0 & 0 & 0 & \textcolor{blue}{1} & 0 & 0 \\
    \rowcolor{lightBlue}
    0 & 1 & 0 & 1 & 0 & \textcolor{blue}{1} & 0 & 0 \\
    0 & 1 & 1 & 0 & 0 & \textcolor{blue}{1} & 0 & 0 \\
    \rowcolor{lightBlue}
    0 & 1 & 1 & 1 & 0 & \textcolor{blue}{1} & 0 & 0 \\
    1 & 0 & 0 & 0 & \textcolor{blue}{1} & 0 & 0 & 0 \\
    \rowcolor{lightBlue}
    1 & 0 & 0 & 1 & \textcolor{blue}{1} & 0 & 0 & 0 \\
    1 & 0 & 1 & 0 & \textcolor{blue}{1} & 0 & 0 & 0 \\
    \rowcolor{lightBlue}
    1 & 0 & 1 & 1 & \textcolor{blue}{1} & 0 & 0 & 0 \\
    1 & 1 & 0 & 0 & \textcolor{blue}{1} & 0 & 0 & 0 \\
    \rowcolor{lightBlue}
    1 & 1 & 0 & 1 & \textcolor{blue}{1} & 0 & 0 & 0 \\
    1 & 1 & 1 & 0 & \textcolor{blue}{1} & 0 & 0 & 0 \\
    \rowcolor{lightBlue}
    1 & 1 & 1 & 1 & \textcolor{blue}{1} & 0 & 0 & 0 \\
    \hline
  \end{tabular}
\end{table}

    \paragraph{Représentation d'un circuit de priorité}
    \mbox{}\vspace{0.5em}

\begin{tikzpicture}[label distance=1mm, scale=0.75]

    \node (x0) at (0,0) {$A_3$};
    \node (x1) at (1,0) {$A_2$};
    \node (x2) at (2,0) {$A_1$};
    \node (x3) at (3,0) {$A_0$};



    \node at ($(x3)+(2,-2)$) (And1) {$Y_3$};
    \node[and gate US, draw, logic gate inputs=nn] at ($(And1)+(0,-1)$) (And2) {};
    \node[and gate US, draw, logic gate inputs=nnn] at ($(And2)+(0.15,-1)$) (And3) {};
    \node[and gate US, draw, logic gate inputs=nnnn] at ($(And3)+(0.15,-1.25)$) (And4) {};


    \draw (x0) |- (And1);
    \draw (x0) |- (And1);
    \draw (And2.input 1) circle (2.5pt);
    
    \draw (x0) |-  ($(And2.input 1)+(-0.1,0)$);
    \draw (x1) |-  (And2.input 2);


    \draw (x0) |-  ($(And3.input 1)+(-0.1,0)$);
    \draw (x1) |-  ($(And3.input 2)+(-0.1,0)$);
    \draw (x2) |-  (And3.input 3);
    \draw (And3.input 1) circle (2.5pt);
    \draw (And3.input 2) circle (2.5pt);


    \draw (x0) |-  ($(And4.input 1)+(-0.1,0)$);
    \draw (x1) |-  ($(And4.input 2)+(-0.1,0)$); 
    \draw (x2) |-  ($(And4.input 3)+(-0.1,0)$);
    \draw (x3) |-  (And4.input 4);
    \draw (And4.input 1) circle (2.5pt);
    \draw (And4.input 2) circle (2.5pt);
    \draw (And4.input 3) circle (2.5pt);

    \draw ($(And2.output) + (0.5,0)$)  node[right] {$Y_2$}; 
    \draw ($(And3.output) + (0.25,0)$) node[right]{$Y_1$};
    \draw (And4.output)  node[right] {$Y_0$};

\end{tikzpicture}


\paragraph{Entrée don't care ou $\mathbb{X}$}
    Ces entrés sont utilisées pour spécifier que la variables possédant le 
    \textit{don't care} n'affecte pas le résultat de la fonction logique.   
    Une file de priorité peut être résumée par la table suivante. 


\begin{table}[H]
  \centering
  \renewcommand{\arraystretch}{1.5}
  \setlength{\arrayrulewidth}{0.4pt}
  \arrayrulecolor{blue}
  \scriptsize
  \begin{tabular}{|c|c|c|c||c|c|c|c|}
    \hline
    \rowcolor{lightBlue}
    \textcolor{myb}{$A_3$} & \textcolor{myb}{$A_2$} & \textcolor{myb}{$A_1$} & \textcolor{myb}{$A_0$} & \textcolor{myb}{$Y_3$} & \textcolor{myb}{$Y_2$} & \textcolor{myb}{$Y_1$} & \textcolor{myb}{$Y_0$} \\
    \hline
    \hline
    0 & 0 & 0 & 0 & 0 & 0 & 0 & 0 \\
    \rowcolor{lightBlue}
    0 & 0 & 0 & 1 & 0 & 0 & 0 & \textcolor{blue}{1} \\
    \rowcolor{lightBlue}
    0 & 0 & 1 & \textcolor{red}{\textit{d}}   & 0 & 0 & \textcolor{blue}{1} & 0 \\
    0 & 1 & \textcolor{red}{\textit{d}} & \textcolor{red}{\textit{d}} & 0 & \textcolor{blue}{1} & 0 & 0 \\
    \rowcolor{lightBlue}
    1 & \textcolor{red}{\textit{d}} & \textcolor{red}{\textit{d}} & \textcolor{red}{\textit{d}} & \textcolor{blue}{1} & 0 & 0 & 0 \\
    \hline
  \end{tabular}
\end{table}


    \paragraph{Contention : signal X }
    Se produit lorsque les portes logiques et les entrées sont telles 
    que la sortie à générer est \textbf{contradictoire}.   


    \begin{center}
        \begin{circuitikz}[label distance=1mm, scale=0.75]
            % Define the NOT gates
            \node[not port] (not1) at (0,0) {};
            \node[not port] (not2) at (0,-2) {};
            \node (jonction) at (1,-1) {};

            % Connect the NOT gates outputs to the same node (contention)
            \draw (not1.out) -| (1,-1) node[above right] {$Y=\mathbb{X}$};    
            \draw (not2.out) -| (1,-1);
            % Connect the NOT gates inputs to the labels
            \draw (not1.in) -- ++(-1,0) node[left] {$A = 1$};
            \draw (not2.in) -- ++(-1,0) node[left] {$B = 0$};
        \end{circuitikz}        
    \end{center}



    \paragraph{Tampon à trois états : signal $\mathbb{Z}$}
    Circuit dans lequel une entrée $\mathbb{E}$ est connecté à une porte tampon et contrôle 
    la \textbf{propagation du signal}. Lorsque l'entrée $\mathbb{E}$ est 
    sous tension haute, la porte agit comme un tampon normal; lorsque l'entrée 
    $\mathbb{E}$ est sous tension basse, la porte produit le signal $\mathbb{Z}$ qui 
    indique que $A$ est \textit{contrôlé}.   

    \begin{center}
        \begin{circuitikz}[scale=0.5]
        % Define buffer gate
        \node[buffer] (buf) at (0,0) {};
        
        % Draw input and output lines
        \draw (buf.in) -- ++(-1,0) node[left] {$A$};
        \draw (buf.out) -- ++(1,0) node[right] {$Y$};
        
        % Draw enable line
        \draw ($(buf.in) + (1, 1)$) -- ++(0,1) node[above] {$E$};
        \end{circuitikz}        
    \end{center}


    \paragraph{Méthodes de Karnaugh}
    Méthode \textbf{graphique}   
    permettant de simplifier les formules de circuits 

    \begin{itemize}
        \item[$\rhd$]   \textbf{Organiser}  les éléments en grille 
            de façon à ce que chaque cellule ne diffère d'une 
            cellule voisine que par \textbf{1 bit}.   
        \item[$\rhd$] Remplir la grille de façon à refléter le 
            \textbf{tableau d'origine}.   
        \item[$\rhd$] \textbf{Grouper} ou entourer les cellules 
            adjacentes qui possèdent un \textbf{1}.   
    \end{itemize}   


    \begin{table}[H]
    \centering
    \footnotesize
    \begin{tabular}{|c|c|c||c|}
    \hline
    \rowcolor{lightBlue}
    \textcolor{myb}{$A$} & \textcolor{myb}{$B$} 
                       & \textcolor{myb}{$C$} 
                       & \textcolor{myb}{$Y$}\\
    \hline
    \hline
    0 & 0 & 0 & \textcolor{blue}{1}   \\ 
    \rowcolor{lightBlue}
    0 & 0 & 1 & \textcolor{blue}{1} \\ 
    \rowcolor{lightBlue}
    0 & 1 & 0 & 0 \\
    0 & 1 & 1 & 0 \\
    \rowcolor{lightBlue}
    1 & 0 & 0 & 0 \\
    1 & 0 & 1 & 0 \\
    \rowcolor{lightBlue}
    1 & 1 & 0 & 0 \\
    1 & 1 & 1 & 0 \\
    \hline
  \end{tabular}
\end{table}        


    \begin{center}
        \begin{karnaugh-map}[4][2][1][$B$][$A$][$C$][$$]
            \minterms{0,4}
            \autoterms[0] 
            \implicant{0}{4}
        \end{karnaugh-map}
    \end{center}


    \begin{center}
        \begin{karnaugh-map}[4][2][1][$B$][$A$][$C$][$$]
          \terms{0}{\tiny{$\overline{A}\cdot\overline{B}\cdot\overline{C}$}}
            \terms{4}{\tiny{$\overline{A}\cdot\overline{B}C$}}
            \autoterms[0]
        \end{karnaugh-map}
    \end{center}
    \textbf{Solution} : considérer les variables qui ne changent pas 
    leurs valeurs entre les cases \textbf{groupés}. Ici, la variable 
    $C$ change de valeur entre le cellule 1 et la cellule 2; 
    elle n'est donc pas considérée dans \textbf{l'équation simplifiée}. 
    Nous avons alors :
    \[Y = \overline{A}\overline{B} \]

    \paragraph{Autre exemple de Karnaugh-map de 3 entrées}
    Parfois, il y a plusieurs \textbf{implicants}. 
    \begin{table}[H]
    \centering
    \footnotesize
    \begin{tabular}{|c|c|c||c|}
    \hline
    \rowcolor{lightBlue}
    \textcolor{myb}{$A$} & \textcolor{myb}{$B$} 
                       & \textcolor{myb}{$C$} 
                       & \textcolor{myb}{$Y$}\\
    \hline
    \hline
    0 & 0 & 0 & 0 \\ 
    \rowcolor{lightBlue}
    0 & 0 & 1 & 0 \\ 
    \rowcolor{lightBlue}
    0 & 1 & 0 & \textcolor{blue}{1} \\
    0 & 1 & 1 & \textcolor{blue}{1} \\
    \rowcolor{lightBlue}
    1 & 0 & 0 & 0 \\
    1 & 0 & 1 & 0 \\
    \rowcolor{lightBlue}
    1 & 1 & 0 & \textcolor{blue}{1} \\
    1 & 1 & 1 & 0 \\
    \hline
  \end{tabular}
\end{table}        


    \begin{center}
      \tiny
        \begin{karnaugh-map}[4][2][1][$B$][$A$][$C$][$$]
            \minterms{1,3,5}
            \autoterms[0] 
            \implicant{1}{3}
            \implicant{1}{5}
        \end{karnaugh-map}
    \end{center}
    \[ Y = \overline{A}B + B\overline{C} \] 


     \paragraph{Autre exemple de Karnaugh-map de 3 entrées}
     Pour les tables de Karnaugh à 4 entrées et plus, les implicants 
     devinnent plus complexes. 

  \begin{table}[H]
    \centering
    \renewcommand{\arraystretch}{1.15}
    \setlength{\arrayrulewidth}{0.4pt}
    \arrayrulecolor{blue}
    \scriptsize
    \begin{tabular}{|c|c|c|c||c|}
      \hline
      \rowcolor{lightBlue}
      \textcolor{myb}{$A$} & \textcolor{myb}{$B$} & \textcolor{myb}{$C$} & \textcolor{myb}{$D$} & \textcolor{myb}{$Y$} 
      \\ \hline
      0 & 0 & 0 & 0 & \textcolor{blue}{1} \\
      \rowcolor{lightBlue}
      0 & 0 & 0 & 1 & 0 \\
      0 & 0 & 1 & 0 &  \textcolor{blue}{1}  \\
      \rowcolor{lightBlue}
      0 & 0 & 1 & 1 & \textcolor{blue}{1} \\
      0 & 1 & 0 & 0 & 0 \\
      \rowcolor{lightBlue}
      0 & 1 & 0 & 1 & \textcolor{blue}{1}  \\
      0 & 1 & 1 & 0 & \textcolor{blue}{1}  \\
      \rowcolor{lightBlue}
      0 & 1 & 1 & 1 & \textcolor{blue}{1} \\
      1 & 0 & 0 & 0 & \textcolor{blue}{1} \\
      \rowcolor{lightBlue}
      1 & 0 & 0 & 1 & \textcolor{blue}{1} \\
      1 & 0 & 1 & 0 & \textcolor{blue}{1} \\
      \rowcolor{lightBlue}
      1 & 0 & 1 & 1 & 0 \\
      1 & 1 & 0 & 0 & 0 \\
      \rowcolor{lightBlue}
      1 & 1 & 0 & 1 & 0 \\
      1 & 1 & 1 & 0 & 0 \\
      \rowcolor{lightBlue}
      1 & 1 & 1 & 1 & 0 \\
      \hline
    \end{tabular}
  \end{table}


  \begin{center}
      \tiny
        \begin{karnaugh-map}[4][4][1][$B$][$A$][$D$][$C$]
            \minterms{0, 2, 5, 6, 12, 13, 8, 9, 10}
            \implicant{2}{6}
            \implicantcorner
            \implicant{12}{9}
            \implicant{5}{13}
        \end{karnaugh-map}
  \end{center}


  \paragraph{Karnaugh-map avec Don't Cares}
  Les \textit{don't care} peuvent complexifier la simplifiation de la 
  fontion à cause du plus grand nombre de \textbf{cas possibles}
  lors du regroupement. 



  \begin{table}[H]
    \centering
    \renewcommand{\arraystretch}{1.5}
    \setlength{\arrayrulewidth}{0.4pt}
    \arrayrulecolor{blue}
    \scriptsize
    \begin{tabular}{|c|c|c|c||c|}
      \hline
      \rowcolor{lightBlue}
      \textcolor{myb}{$A$} & \textcolor{myb}{$B$} & \textcolor{myb}{$C$} & \textcolor{myb}{$D$} & \textcolor{myb}{$Y$} 
      \\ \hline
      0 & 0 & 0 & 0 & \textcolor{blue}{1} \\
      \rowcolor{lightBlue}
      0 & 0 & 0 & 1 & 0 \\
      0 & 0 & 1 & 0 &  \textcolor{blue}{1}  \\
      \rowcolor{lightBlue}
      0 & 0 & 1 & 1 & \textcolor{blue}{1} \\
      0 & 1 & 0 & 0 & 0 \\
      \rowcolor{lightBlue}
      0 & 1 & 0 & 1 & \textcolor{red}{$\mathbb{X}$}    \\
      0 & 1 & 1 & 0 & \textcolor{blue}{1}  \\
      \rowcolor{lightBlue}
      0 & 1 & 1 & 1 & \textcolor{blue}{1} \\
      1 & 0 & 0 & 0 & \textcolor{blue}{1} \\
      \rowcolor{lightBlue}
      1 & 0 & 0 & 1 & \textcolor{blue}{1} \\
      1 & 0 & 1 & 0 & \textcolor{blue}{1} \\
      \rowcolor{lightBlue}
      1 & 0 & 1 & 1 & \textcolor{red}{$\mathbb{X}$} \\
      1 & 1 & 0 & 0 & \textcolor{red}{$\mathbb{X}$} \\
      \rowcolor{lightBlue}
      1 & 1 & 0 & 1 & \textcolor{red}{$\mathbb{X}$} \\
      1 & 1 & 1 & 0 & \textcolor{red}{$\mathbb{X}$} \\
      \rowcolor{lightBlue}
      1 & 1 & 1 & 1 & \textcolor{red}{$\mathbb{X}$} \\
      \hline
    \end{tabular}
  \end{table}

  


  
  % paragraph  (end)


  \begin{center}
      \tiny
        \begin{karnaugh-map}[4][4][1][$B$][$A$][$D$][$C$]
            \minterms{0,2,6, 12, 13, 8,9}
            \terms{3,5,7, 15, 14, 13, 10}{$\mathbb{X}$}
            \implicantcorner
            \implicant{3}{10}
            \implicant{12}{10}
        \end{karnaugh-map}
  \end{center}

  \[ Y = A + \overline{B} \cdot \overline{D} + C \]


  \paragraph{Karnaugh-map de 5 entrées}
  Il faut considérer \textbf{deux} K-map de \textbf{quatre} variables. Par convention, 
  on peut omettre la première variable dans les deux K-map; on considère que la 
  dans la première K-map, la variable irgnorée a une valeur de \textbf{0} et, dans la 2e K-map, 
  elle a une valeur de \textbf{1}. 
  \href{https://www.youtube.com/watch?v=CZPwYZdmMI0&t=417s}{\texttt{Exemple sur Youtube}}. 
  Soit la fonction et K-maps correspondantes :  
  \begin{align*}
    f(A, B, C, D, E) = &\sum(0, 1, 2, 4, 5, 6, 10, 13, 
                    \\ &14 18, 21, 22, 24, 26, 29, 30)
  \end{align*}


  \begin{figure}[H]  
    \caption*{\footnotesize{K-map de $BCDE$ en considérant $A = 0$}}
    \begin{center}
        \begin{karnaugh-map}[4][4][1][$C$][$B$][$E$][$D$]
          \terms{0}{1\;\tiny{\textcolor{myp}{0}}}
          \terms{1}{1\;\tiny{\textcolor{myp}{1}}}
          \terms{4}{1\;\tiny{\textcolor{myp}{4}}}
          \terms{5}{1\;\tiny{\textcolor{myp}{5}}}
          \terms{7}{1\;\tiny{\textcolor{myp}{7}}}
          \terms{8}{1\;\tiny{\textcolor{myp}{8}}}
          \terms{9}{1\;\tiny{\textcolor{myp}{9}}}
          \terms{10}{1\;\tiny{\textcolor{myp}{10}}}
          \terms{11}{1\;\tiny{\textcolor{myp}{11}}}
          \terms{2}{0\;\tiny{\textcolor{myp}{2}}}
          \terms{3}{0\;\tiny{\textcolor{myp}{3}}}
          \terms{6}{0\;\tiny{\textcolor{myp}{6}}}
          \terms{12}{0\;\tiny{\textcolor{myp}{12}}}
          \terms{13}{0\;\tiny{\textcolor{myp}{13}}}
          \terms{14}{0\;\tiny{\textcolor{myp}{14}}}
          \terms{15}{0\;\tiny{\textcolor{myp}{15}}}
            \minterms{0, 1, 4, 5, 7, 8, 9, 11, 10}
            \implicant{0}{5}
            \implicant{5}{7}
            \implicant{8}{10}
        \end{karnaugh-map}    \end{center}
  \end{figure}

  \[ \textcolor{myyellow!50}{D\overline{E}} + \textcolor{green}{C\overline{D} \cdot E}   
  + \textcolor{red}{\overline{A} \cdot \overline{B} \cdot \overline{D}}  \]
    


   \begin{figure}[H]
    \caption*{\footnotesize{K-map de $BCDE$ en considérant $A = 1$}}
    \begin{center}
    
      \begin{karnaugh-map}[4][4][1][$C$][$B$][$E$][$D$]
          \terms{0}{0\;\tiny{\textcolor{myp}{16}}}

          \terms{1}{0\;\tiny{\textcolor{myp}{17}}}
          
          \terms{4}{0\;\tiny{\textcolor{myp}{20}}}
          \terms{5}{1\;\tiny{\textcolor{myp}{21}}}
          \terms{7}{1\;\tiny{\textcolor{myp}{23}}}


          \terms{8}{1\;\tiny{\textcolor{myp}{24}}}
          \terms{9}{1\;\tiny{\textcolor{myp}{25}}}
          \terms{10}{1\;\tiny{\textcolor{myp}{26}}}
          \terms{11}{1\;\tiny{\textcolor{myp}{27}}}
          \terms{2}{0\;\tiny{\textcolor{myp}{18}}}
          \terms{3}{0\;\tiny{\textcolor{myp}{19}}}
          \terms{6}{0\;\tiny{\textcolor{myp}{22}}}

          \terms{12}{0\;\tiny{\textcolor{myp}{28}}}
          \terms{13}{0\;\tiny{\textcolor{myp}{29}}}
          \terms{14}{0\;\tiny{\textcolor{myp}{28}}}
          \terms{15}{0\;\tiny{\textcolor{myp}{31}}}
            \minterms{2, 5, 7, 8, 9, 10, 11}
            \implicantedge{2}{2}{10}{10}
            \implicant{5}{7}
            \implicant{8}{10}
        \end{karnaugh-map}    \end{center}
    \caption{}
  \end{figure}


  \[ \textcolor{myyellow!50}{D\overline{E}} + \textcolor{green}{C\overline{D} \cdot E}   
  + \textcolor{red}{A B \cdot \overline{C}}  \]

  \paragraph{Karnaugh-map de 6 entrées}
  \mbox{}\vspace{0.5em}
  \begin{note}{}{}
        On peut utiliser la même approches que la méthode pour entrée, cette fois 
        en considérant \textbf{4} K-map de 4 variables superposées dans un cube 
        tridimensionnel
        \href{https://www.youtube.com/watch?v=LXJXZOqZpGk}{Exemple sur \texttt{Youtube}}
  \end{note}
    \paragraph{Multiplexeur}
    \begin{itemize}
        \item[$\rhd$ ]  $2^n$ lignes d'entrées 
        \item[$\rhd$ ]  $2^n$ N lignes de sélections 
        \item[$\rhd$ ]  $2^n$ Une seule sorties $Y$  
    \end{itemize}
    Possède deux implémentations secondaires, soit (1) \textbf{portes logiques}
    et (2) \textbf{tampons} à trois états. 
    

    \paragraph{Quine-McCluskey Method} 
    Méthode tabulaire qui permet de réduire les expression \textbf{SOP}  


    \begin{center}
       \begin{tabular}{|c|c|c|c|c|c|}
      \hline
      \rowcolor{myg}
       &  \textcolor{white}{A} & \textcolor{white}{B} & \textcolor{white}{C} &
      \textcolor{white}{D} & \textcolor{white}{F}   \\
      \hline
      \textcolor{myb}{0} & 0 & 0 & 0 & 0 & \textcolor{myb}{$d$} \\
      \rowcolor{myg!40} 1 & 0 & 0 & 0 & 1 & 1 \\
      \textcolor{red}{2} & 0 & 0 & 1 & 0 & 0 \\
      \rowcolor{myg!40} \textcolor{red}{3} & 0 & 0 & 1 & 1 & 1 \\
      4 & 0 & 1 & 0 & 0 & 0 \\
      \rowcolor{myg!40} \textcolor{red}{5} & 0 & 1 & 0 & 1 & 1 \\
      \textcolor{red}{6} & 0 & 1 & 1 & 0 & 1 \\
      \rowcolor{myg!40} \textcolor{red}{7} & 0 & 1 & 1 & 1 & 1 \\
      8 & 1 & 0 & 0 & 0 & 0 \\
      \rowcolor{myg!40} 9 & 1 & 0 & 0 & 1 & 0 \\
      \textcolor{red}{10} & 1 & 0 & 1 & 0 & 1 \\
      \rowcolor{myg!40} \textcolor{myb}{11} & 1 & 0 & 1 & 1 & \textcolor{myb}{$d$} \\
      12 & 1 & 1 & 0 & 0 & 0 \\
      \rowcolor{myg!40} \textcolor{red}{13} & 1 & 1 & 0 & 1 & 1 \\
      14 & 1 & 1 & 1 & 0 & 0 \\
      \rowcolor{myg!40} \textcolor{myb}{15} & 1 & 1 & 1 & 1 & \textcolor{myb}{$d$} \\
      \hline 
      \end{tabular}  
    \end{center}

  \begin{align*}
    \textcolor{red}{\sum (1, 3, 5, 6, 7, 10, 13)} + \textcolor{myb}{\sum_{d} (0, 11, 15)}  
  \end{align*}

  \textbf{Formation du 1er tableau}
  \begin{itemize}
    \item [$\rhd$ ] Considérer les mintermes tels que $F \neq 0$ 
    \item [$\rhd$ ] Grouper les mintermes : 
      \begin{itemize}
        \item [1.] Minterme contiennent aucun 1 
        \item [2.] Mintermes contiennent un 1 
        \item [ ] $\dots$ 
        \item [4.] Minterme contient quatre 1 


      \end{itemize}
  \end{itemize}


      \begin{center}
          \begin{tabular}{c c c c}
          \rowcolor{myg}
          \textcolor{white}{A} & \textcolor{white}{B} & \textcolor{white}{C} & \textcolor{white}{D} \\
          \hline
          0 & 0 & 0 & 0 \\
          \arrayrulecolor{red}
          \hline
          \rowcolor{myg!40}
          0 & 0 & 0 & 1 \\
          \hline
          0 & 0 & 1 & 1 \\
          \arrayrulecolor{myg}
          0 & 1 & 0 & 1 \\
          0 & 1 & 1 & 0 \\
          1 & 0 & 1 & 0 \\
          \arrayrulecolor{red}
          \hline
          \rowcolor{myg!40}
          0 & 1 & 1 & 1 \\
          \rowcolor{myg!40}
          1 & 0 & 1 & 1 \\
          \rowcolor{myg!40}
          1 & 1 & 0 & 1 \\
          \hline
          1 & 1 & 1 & 1 \\ 
          \arrayrulecolor{myg!40}
          \end{tabular}  
      \end{center} 


  \textbf{Formation de la première tablea réduite}  
  \begin{itemize}
    \item [$\rhd$ ] Comparer les groupes adjacents. 
    \item [$\rhd$ ] Trouver les termes qui diffèrent d'une valeur booléenne. 
    \item [$\rhd$ ] Lorsqu'un différence de $1$ seul terme est trouvée, 
      placer le terme réduit dans la nouvelle terme avec un \textbf{underscore} 
      \texttt{"-"} à la position de la différence.    
    \item [$\rhd$ ] Comparer chaque terme d'un groupe à chaque terme de 
      son groupe adjacent. 
  \end{itemize}


      \begin{center}
          \begin{tabular}{c c c c c}
          \rowcolor{myg}
          \textcolor{white}{A} & \textcolor{white}{B} & \textcolor{white}{C} & \textcolor{white}{D} & \\
          \hline  
          0 & 0 & 0 & \texttt{\_}  & * \\
          \arrayrulecolor{red}
          \hline
          0 & 0 & \texttt{\_}   & 1 & \\
          0 & \texttt{\_} & 0  & 1 & \\
          \hline
          0 & \texttt{\_}   & 1 & 1 & \\
          \texttt{\_}   & 0 & 1 & 1 & \\
          0 & 1 & \texttt{\_}   & 1 & \\
          \texttt{\_}   & 1 & 0 & 1 & \\
          0 & 1 & 1 & \texttt{\_}  & * \\
          1 & 0 & 1 & \texttt{\_}   & * \\
          \hline
          \texttt{\_}   & 1 & 1 & 1 & \\ 
          1 & \texttt{\_}   & 1 & 1 & \\ 
          1 & 1 & \texttt{\_}   & 1 & \\ 
          \end{tabular}  
      \end{center} 

      
      \noindent \textbf{Formation de 2e tableau} \\ 
      On continue la comparaison. Si pour un groupe donné, un terme ne peut pas 
      être comparé avec \textit{aucun groupe adjacent}, il s'agit d'un implicant premier. 
      Par exemple, le premier terme du tableau précédent ne peut être comparé 
      avec aucun terme de du groupe qui lui est adjacent. 


\begin{center}
          \begin{tabular}{c c c c c}
          \rowcolor{myg}
          \textcolor{white}{A} & \textcolor{white}{B} & \textcolor{white}{C} & \textcolor{white}{D} &  \\
          \hline  
          0 & \texttt{\_} & \texttt{\_} & 1 &  * \\
          \arrayrulecolor{red}
          \hline
          \texttt{\_}   & \texttt{\_}   & 1 & 1 &  *\\
          \texttt{\_}   & 1 & \texttt{\_}  & 1 & * \\ 
        \end{tabular}
      \end{center} 


\noindent\textbf{Formation de la table de choix}  
    \begin{center}
        \scalebox{0.75}{
        \begin{tabular}{|c|c|c|c|c|c|c|c|}
        \hline
        \multirow{2}{*}{IP} & \multicolumn{7}{c|}{Minterms} \\ \cline{2-8} 
        & 0001 & 0011 & 0101 & 0110 & 0111 & 1010 & 1101 \\ \hline
        000-  & $\checkmark$ & & & & & & \\ \hline
        011-   & & &  & \textcolor{black}{$\checkmark$}  & $\checkmark$ & & \\ \hline
        101-   & & & & & & \textcolor{black}{$\checkmark$} & \\ \hline
        0--1 & $\checkmark$ & $\checkmark$ & $\checkmark$ & & $\checkmark$ & & \\ \hline
        --11 & & $\checkmark$ & & & $\checkmark$ & & \\ \hline
        -1-1 & & & $\checkmark$ & & $\checkmark$ & & $\checkmark$ \\ \hline
        \end{tabular}}
    \end{center}

    \noindent \textbf{Faire le choix}  \\ 
    Considérer un implicant premier à la fois et  
    cocher les mintermes qui sont couverts par l'implicant premier. 
    Chercher ensuite les \textbf{colonnes} où il y a un seul cochet; l'implicant 
    premier à cet endroit est donc un \textbf{implicant obligatoire}  


    \begin{center}
        \scalebox{0.75}{
        \begin{tabular}{|c|c|c|c|c|c|c|c|}
        \hline
        \multirow{2}{*}{IP} & \multicolumn{7}{c|}{Minterms} \\ \cline{2-8} 
        & 0001 & 0011 & 0101 & 0110 & 0111 & 1010 & 1101 \\ \hline
        000\texttt{\_}  & $\checkmark$ & & & & & & \\ \hline
        011\texttt{\_}\textcolor{red}{*}  & & &  & \textcolor{red}{$\checkmark$}  
                                          & $\checkmark$ & & \\ \hline
        101\texttt{\_}    \textcolor{red}{*} & & & & & & \textcolor{red}{$\checkmark$} & \\ \hline
        0 \texttt{\_\_}1 & $\checkmark$ & $\checkmark$ & $\checkmark$ & & $\checkmark$ & & \\ \hline
        \texttt{\_\_}11 & & $\checkmark$ & & & $\checkmark$ & & \\ \hline
        \texttt{\_}1\texttt{\_}1 \textcolor{red}{*}  & & & $\checkmark$ & & $\checkmark$ & & \textcolor{red}{$\checkmark$} \\ \hline
        \end{tabular}}
    \end{center}
    \noindent 
    Regarder ensuite quels mintermes sont couvert pas les 
    \textbf{implicant obligatoire}. Dans l'exemple donnée les 3 implicants obligatoire 
    couvrent 5 des 7 mintermes. 

    Parmi les implicants restants, il faut choisir la quantité minimale d'implicants 
    premiers qui permettent de couvrir les mintermes restants. 



    \begin{center}
        \scalebox{0.75}{
        \begin{tabular}{|c|c|c|c|c|c|c|c|}
        \hline
        \multirow{2}{*}{IP} & \multicolumn{2}{c|}{Minterms} \\ \cline{2-3} 
        & 0001 & 0011 \\ \hline
        000\texttt{\_}   & $\checkmark$ & \\ \hline
        0 \texttt{\_\_}1   & \textcolor{red}{$\checkmark$}  
                          & \textcolor{red}{$\checkmark$} \\ \hline
        101\texttt{\_}  & & $\checkmark$ \\ \hline
        \end{tabular}}
    \end{center}

    \noindent 
    \textbf{Écrire l'équation finale}  
    Considérer les implicants obligatoires trouvés et déduire la valeur des entrées 
    formant la SOP en fonction du code de l'implicant obligatoire. 
    \begin{align*}
      f(A, B, C, D) &= \texttt{011\_ + 101\_ + \_1\_1 + \_1\_1} \\ 
                    &= BC + A\overline{B}C + BD + \overline{A}D
    \end{align*}

    \paragraph{Multiplexeur MUX}
    \begin{itemize}
      \item [$\rhd$ ] Possède $2^n$ lignes d'entrée : $D_0, D_1, \dots, D_{2^n}$
      \item [$\rhd$ ] Une seule sortie $Y$ 
      \item[$\rhd$ ] \textbf{Rôle} : propager sur la sortie la valeur de l'une des $2^n$ entrées.   
    \end{itemize}


    \begin{table}[H]
      \begin{center}
        \renewcommand{\arraystretch}{1.15}
        \fontfamily{flr}\selectfont
        \footnotesize
            \begin{tabular}{l l l |l}
            \arrayrulecolor{blue}\hline
            \rowcolor{lightBlue}
            \textcolor{myb}{S} & \textcolor{myb}{$D_0$} &  
            \textcolor{myb}{$D_1$} & \textcolor{myb}{$Y$}
            \\
            \hline
            \arrayrulecolor{black}
            0 & 0 & 0 & 0 
            \\
            \hline
            0 & 0 & 1 & 1   
            \\
            \hline
            0 & 1 & 0 & 0 
            \\ 
            \hline 
            0 & 1 & 1 & 1   
            \\ 
            \hline 
              1 & 0 & 0 & 0 
            \\ 
            \hline           
             1 & 0 & 1 & 0 
            \\ 
            \hline            
            1 & 1 & 0 & 1 
            \\ 
            \hline 
            1 & 1 & 1 & 1 
            \\ 
            \hline 
                \end{tabular}
    \end{center}
    \end{table}

    \begin{center}
        \begin{karnaugh-map}[4][2][1][$D_1D_0$][$S$]
            \minterms{2, 3, 5, 7}
            \implicant{3}{2}
            \implicant{5}{7}
        \end{karnaugh-map} 
    \end{center}

    \[ Y = D_0\overline{S} + D_1S \] 


    \begin{figure}[H]
      \begin{center}
        \includegraphics[width=0.15\textwidth]{ImplemMUX.png}
      \end{center}
    \end{figure}

    \noindent\textbf{Logique en utilisant MUX} \\   
    Soit une \textit{fonction logique} $f(A, B...)$   
    dépendant de $2^n$ entrées, on peut utiliser un MUX pour 
    repréter cette fonction.  


    \begin{table}[H]
      \begin{center}
        \renewcommand{\arraystretch}{1.15}
        \fontfamily{flr}\selectfont
        \footnotesize
            \begin{tabular}{l l|l}
            \arrayrulecolor{blue}\hline
            \rowcolor{lightBlue}
            \textcolor{myb}{A} & \textcolor{myb}{B} & 
            \textcolor{myb}{Y}  
            \\
            \hline
            \arrayrulecolor{black}
            0 & 0 & 0 
            \\
            \hline
            0 & 1 & 0  
            \\
            \hline
            1 & 0 & 0
            \\ 
            \hline 
            1 & 1 & 1 
            \\
            \hline
                \end{tabular}
    \end{center}
    \end{table}


    \begin{figure}[H]
      \begin{center}
        \includegraphics[width=0.15\textwidth]{MUXExemple.png}
      \end{center}
    \end{figure}

    \begin{note}{}{}
      Sur les schéma de circuit, le \textbf{triangle} $\rhd$ représent la valeur source $0$ 
        et la \textbf{barrre} est $1$  
    \end{note}

    \noindent \textbf{Réduction de la taile de MUX}  
    \begin{figure}[H]
      \begin{center}
        \includegraphics[width=0.28\textwidth]{ReductionMUX.png}
      \end{center}
    \end{figure} 

    \paragraph{Décodeur}
    \begin{itemize}
      \item [$\rhd$ ] Possède $N$ entrées et $2^N$ sorties 
      \item [$\rhd$ ] Les sorties expriment \textit{combinaisons binaires} possibles des E.    
     
    \end{itemize}


    \begin{table}[H]

      \begin{center}
        \renewcommand{\arraystretch}{1.5}
        \fontfamily{flr}\selectfont
        \footnotesize
            \begin{tabular}{l l l l l l}
            \arrayrulecolor{blue}\hline
            \rowcolor{lightBlue}
            \textcolor{myb}{$A_1$} & \textcolor{myb}{$A_0$} &
            \textcolor{myb}{$Y_3$} & \textcolor{myb}{$Y_2$} &
            \textcolor{myb}{$Y_1$} & \textcolor{myb}{$Y_0$} 
            \\
            \hline
            \arrayrulecolor{black}
            0 & 0 & 0 & 0 & 0 & \textcolor{red}{1}  
            \\
            \hline
            0 & 1 & 0 & 0 & \textcolor{red}{1} & 0
            \\
            \hline
            1 & 0 & 0 & \textcolor{red}{1} & 0 & 0
            \\
            \hline
            1 & 1 & \textcolor{red}{1} & 0 & 0 & 0
            \\
            \hline
                \end{tabular}
    \end{center}
    \end{table}
    \noindent \textbf{Implémentation de décodeur}  


    \begin{center}
    \begin{tikzpicture}
        % Draw the box for the decoder
        \draw (0,0) rectangle (3,4);
        \node at (1.5,2) {Décodeur $2 \colon 4$ };

        % Draw input lines and labels
        \draw (-1,3.5) -- (0,3.5) node[at start, left] {$A_1$};
        \draw (-1,2.5) -- (0,2.5) node[at start, left] {$A_0$};

        % Draw output lines and labels
        \draw (3,3) -- (4,3) node[at end, right] {$Y_3$};
        \draw (3,2.5) -- (4,2.5) node[at end, right] {$Y_2$};
        \draw (3,1.5) -- (4,1.5) node[at end, right] {$Y_1$};
        \draw (3,1) -- (4,1) node[at end, right] {$Y_0$};

        % Draw the binary numbers next to output lines
        \node at (3.5,3) [red, anchor=south] {11};
        \node at (3.5,2.5) [red, anchor=south] {10};
        \node at (3.5,1.5) [red, anchor=south] {01};
        \node at (3.5,1) [red, anchor=south] {00};
    \end{tikzpicture}
    \end{center}



    \begin{figure}[H]
      \begin{center}
        \includegraphics[width=0.20\textwidth]{ImplementationDecodeur.png}
      \end{center}
    \end{figure}

    \paragraph{Caractéristiques électriques et tempo.}
    Les changements $0 \rightarrow 1$ ou $1 \rightarrow 0$ \textbf{ne sont pas}  
    immédiat. Par exemple, s'il y a une porte tampons qui 
    propage la valeur d'une entrée $A$, il y a un délais 
    pour que $Y$ passe de l'état précédent au nouvel état acuel 
    de $A$. 

    \noindent \textbf{Détails de propagation $t_{pd}$}  
    Délais maximal; il est évalué du premier changement de l'entrée 
    jusqu'à la \textit{stabilisation} de la sortie.   


   \noindent \textbf{Détails de contamination$t_{cd}$}  
   Délais minimum; il provient du premier changement de l'entrée 
   jusqu'au premier changement de la sortie. 


   \noindent \textbf{Chemin critique}  
   Chemin le plus long pour que la propagation \textbf{totale} s'effectue. Pour le 
   calculer, il faut additionner les $t_{pd}$ des portes impliquant 
   le plus long chemin. 


   \noindent \textbf{Chemin court}  
   Chemin le plus court engendrant une contamination. Pour le 
   calculer, il faut additionner les $t_{cd}$ des portes impliquant 
   le plus court chemin. 


   \begin{figure}[H]
    \begin{center}
      \includegraphics[width=0.25\textwidth]{Glitch.png}
    \end{center}
   \end{figure}


   \paragraph{Aléas}
   Se produit lorsque des transitions \textit{non prévues} par 
   la fonction booléenne apparaissent suite à une variation 
   du vecteur d'E. 


   \paragraph{Exemple de glitch}
   Puisque le chemin le plus court pour l'entrée $B$ 
   est la 2e porte $\mathbb{ET}$, lorsque $B$ 
   transitionne, on a  $B \colon 1 \rightarrow 0$,
   $1(C)^0(B) \rightarrow 0$. Lors que la porte 
   ou est temporairement dans l'état $0$, on a 
   $0^0 \rightarrow \textcolor{gray}{0}$.   


   \begin{figure}[H]
    \begin{center}
      \includegraphics[width=0.35\textwidth]{ExempleGlitch.png}
    \end{center}
   \end{figure}

   \paragraph{Horloge}
   Les signaux ne se trouve pas toujours dans leur 
   \textit{état valide}, à cause des délais de montée, descente 
   et propagation. Pour éviter les aléas, on impose au 
   système de lire les valeurs à des \textbf{instants précis}  
   et à des intervales réguliers. 

   L'intervale de temps entre deux impulsions régulières 
   est le \textit{temps de cycle} ou \textit{période d'horloge}    


   \begin{figure}[H]
    \begin{center}
      \includegraphics[width=0.13\textwidth]{ExempleHorloge.png}
    \end{center}
   \end{figure}

   \chapter{Circuits séquentiels}

   \paragraph{Principe}
   En plus de l'état des entrées, les circuits 
   logiques considèrent aussi les 
   \textit{information précédemment traitées}; 
   ils possèdent donc une mémoire. 


   \begin{figure}[H]
    \begin{center}
      \includegraphics[width=0.35\textwidth]{exCircsequentiel.png}
    \end{center}
    \caption{Sortie réinjectée dans porte entrée}
   \end{figure}


   \paragraph{Circuit bistable}
   Possède deux inverseurs et \textbf{aucune entrée} 

   \begin{figure}[H]
    \begin{center}
      \includegraphics[width=0.25\textwidth]{CirBistable.png}
    \end{center}
   \end{figure}

   \paragraph{Latch S/R}
   Application d'un circuit bistable contenant 
   additionnelement deux portes $\mathbb{OU}$ 
   et deux entrées $R$ et $S$. En pratique, le 
   Latch \textit{Set Reset} stocke un bit d'état 
   $Q$ et contrôle ce bit par les entrée 
   $S$. Lors de l'étape \textit{Set}, 
   on a $S = 1, R = 0, Q =1 $. Lors de l'étape 
   \textit{Reset}, on a 
   $S = 0, R = 1, Q = 0$. 


   \begin{figure}[H]
    \begin{center}
      \includegraphics[width=0.15\textwidth]{Latch1.png}
    \end{center}
    \caption{\tiny$(R = 0)\land(S = 1) \implies (Q = 1) 
    \land(\overline{Q} = 0)$}
   \end{figure}


   \begin{figure}[H]
    \begin{center}
      \includegraphics[width=0.15\textwidth]{Latch2.png}
    \end{center}
    \caption{\tiny$(R = 1)\land(S = 0) \implies (Q = 0) 
    \land(\overline{Q} = 1)$}
   \end{figure}

   Lorsque $S$ et $R$ ont pour valeur \textbf{0}, 
   il y a formation de mémoire. Pour un cycle on 
   obtient $Q_{prev} = 0$, pour le cycle suivant 
   on obtient $Q_{prev} = 1$

   \begin{figure}[H]
    \begin{center}
      \includegraphics[width=0.25\textwidth]{Latch3.png}
    \end{center}
    \caption{
        \tiny$(R = 0)\land(S = 0) \implies (Q_{prev} = 0$)} 
   \end{figure}


   Lorsque $R$ et $S$ on pour valeur \textbf{1}, 
   il y a une incohérence; $Q = \overline{Q}$. 


   \begin{figure}[H]
    \begin{center}
      \includegraphics[width=0.15\textwidth]{Latch4.png}
    \end{center}
    \caption{
        \tiny$(R = 0)\land(S = 0) \implies (Q_{prev} = 0$)} 
   \end{figure}


   \paragraph{Verrou $D$ }

  
   Circuit dans lequel la valeur de $CLK$ contrôle 
   la valeur propagée par l'entrée $D$. Lorsque 
   le $CLK$ a une valeur de \textbf{0} la 
   valeur de $Q$ varie, indépendamment de 
   $D$. Lorsque la valeur de $CLK$ est \textbf{1}  
   la valeur de $D$ est propagée. 
   okkqwkboo

   \begin{figure}[H]
    \begin{center}
      \includegraphics[width=0.25\textwidth]{LatchD.png}
    \end{center}
    \caption{}
   \end{figure}


   \paragraph{Bascule $D$}
   Circuit dans lequel $D$ est propagé sur le front 
   montant de l'horloge $CLK$. Lorqque $CLK$ passe 
   de \textbf{0} à \textbf{1}, $D$ est propagé 
   vers $Q$. Autrement, $Q$ préserve sa valeur précédente. 

   \begin{figure}[H]
    \begin{center}
      \includegraphics[width=0.25\textwidth]{BasculeD.png}
    \end{center}
   \end{figure}

   \begin{figure}[H]
    \begin{center}
      \includegraphics[width=0.25\textwidth]{VerrouBasculeComparaison.png}
    \end{center}
    \caption{Comparaison Verrou D et Bascule D}
   \end{figure}

   \paragraph{Registre}
    Ils sont composés de bascules. Chaque bascule a un index et la taille 
    du bus de bascule est spécifiée. 
    \begin{itemize}
      \item [$\rhd$ ] Registre $3:0$ possède $4$ bascules.
    \end{itemize} 

    \begin{figure}[H]
      \begin{center}
        \includegraphics[width=0.15\textwidth]{RegistreExemple.png}
      \end{center}
    \end{figure}


    \begin{figure}[H]
      \begin{center}
        \includegraphics[width=0.15\textwidth]{4Bascules}
      \end{center}
    \end{figure}


    \paragraph{Bascule $D$ avec \texttt{Enable}}
    On ajout un signal \texttt{enable}. Est implémenté en ajoutant un MUX 2:1 
    \begin{itemize}
      \item [$\rhd$ ] \texttt{enable} = 1 : $D$ passe vers $Q$ au fron montant $CLK$  
      \item [$\rhd$ ] \texttt{enable} = 0 : Bascule retient sa valeur précédente; 
        l'ancien $Q$ deeure.
    \end{itemize}
    Sert à mémoriser l'information de $Q$ durant plusieurs cyles d'horloge. 


    \begin{figure}[H]
      \begin{center}
        \includegraphics[width=0.28\textwidth]{BasculeDEnable.png}
      \end{center}
    \end{figure}


    \paragraph{Bascule $D$ avec \texttt{reset}}
    On ajout un signal \texttt{reset}. 
    \begin{itemize}
      \item [$\rhd$ ] \texttt{reset} = 1 : $Q$ est fixé à $0$ 
      \item [$\rhd$ ] \texttt{reset} = 1 : Fonctionne comme une bascule 
        $D$ normale
    \end{itemize}
    Peut être synchrone (fixe $Q = 0$ à condition que 
    \texttt{reset = 1} \textbf{et} $CLK = 1$ front montant) ou 
    asynchrone (fixe $Q = 0$ pour \texttt{reset} = 1, 
    indépendamment de $CLK$. 


    \begin{figure}[H]
      \begin{center}
        \includegraphics[width=0.2\textwidth]{BasculeDReset.png}
      \end{center}
    \end{figure}


    \paragraph{Bascule $D$ avec \texttt{set}  }
    On ajout un signal \texttt{set}. 
    \begin{itemize}
      \item [$\rhd$ ] \texttt{set} = 1 : $Q$ est fixé à $1$ 
      \item [$\rhd$ ] \texttt{set} = 0 : Fonctionne comme une bascule 
        $D$ normale
    \end{itemize}


    \begin{figure}[H]
      \begin{center}
        \includegraphics[width=0.2\textwidth]{BasculeDSet.png}
      \end{center}
    \end{figure}

    \paragraph{Automate fini}
    Il s'agit de la \textbf{base théorique} des circuits séquentiels  
    Possède un nombre fini d'\textbf{états}. Pour u coupe d'instants 
    (\texttt{t}, \texttt{t+1}), il possède une réponse $S$, un entre 
    $E$ et un état $Q$. 
    



    



    
    
    
   
   
   
   
   

   

   
   


   







  

    

    
    







    \end{multicols*}
\end{document}




